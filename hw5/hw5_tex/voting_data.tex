\qns{PCA and voting data}
In this problem, we look at Senate voting data. The data is contained in a $n \times m$ data matrix $X$, where each row corresponds to a Senator, and each column to a bill. Each entry of $X$ is either $1,-1$ or  $0$ depending on whether the senator voted for, against or abstained. 
\begin{enumerate}
    \qitem
    We want to assign a score to each senator based on their voting pattern. For this let us pick $a \in \R m$, and a scalar $b$ and define the score for senator $i$ as:
    \begin{align*}
        f(x_i, a, b) =  x_i\tran a + b, \quad i = 1, 2, \hdots, n.
    \end{align*}
    Note that in our notation the rows of $X$ are $x_i\tran$ and thus each $x_i\tran$ is a row vector of length $m$.\\
    
    Let us denote by $f(X, a, b)$, the column vector of length $n$, obtained by stacking the scores for each senator. Then,
    \begin{align*}
        z = f(X, a, b) = Xa + b\textbf{1} \in \mathbb{R}^n
    \end{align*}
    where \textbf{1} is a vector with all entries equal to 1. Let us denote the mean value of $z$ by $\mu_z = \dfrac{1}{n} \textbf{1}^\top z$. Further let $\mu_x\tran$ denote the row vector corresponding to mean of each column of $X$. Then  
    \begin{align*}
        \mu_z &= \dfrac{1}{n}\sum\limits_{i=1}^n f(x_i, a, b)\\
             &= a\tran \mu_x + b. 
    \end{align*}
    Then the empirical variance of the scores can be obtained as:
    \begin{align*}
        Var(f(X,a,b)) &= Var(z) \\
                      &= \frac{1}{n} (z - \mu_z \textbf{1})\tran (z - \mu_z\textbf{1}) \\  
                      &= \frac{1}{n} (Xa + b\textbf{1} - a^\top \mu_x \textbf{1} - b\textbf{1})\tran (Xa + b\textbf{1} - a^\top \mu_x \textbf{1} - b\textbf{1}) \\  
                      &= \dfrac{1}{n} (Xa  -  \textbf{1}\mu_x^\top a )\tran (Xa  - \textbf{1} \mu_x^\top a ) \\
                      &= \frac{1}{n}a\tran (X - \textbf{1}\mu_x\tran)\tran (X - \textbf{1}\mu_x\tran)a
    \end{align*}
    % Note that here we abuse notation a bit in line with what python supports to allow subtraction of matrices and vectors. In such cases the dimensions of the entities involved help us determine what the notation stands for. For example $X - \mu_x\tran$ refers to the mean row vector being subtracted from each row of $X$.
    We observe that the variance of score functions depends on ``centered data'' and does not depend on $b$.\\
    
    For the remainder of the problem we assume that the data has been ``centered'' (i.e  mean of each column of $X$ is zero) and set $b = 0$ so that $\mu_z = 0$.
    
    This leads us to the simpler formula,
    \begin{equation*}
        Var(f(X,a)) = \frac{1}{n}a\tran X\tran X a.
    \end{equation*}
    Suppose we restrict $a$ to have unit-norm. Then,find $a$ that maximizes $Var(f(X,a))$. What is the value of the maximum variance?

    \sol{\input{voting_data_solution/1.1.tex}
    }
    
	\qitem
	From the previous part what can you say about how senators vote as compared to their party average? Compute the variance of the scores and comment on the projections along $a$ for the following two cases:
	\begin{enumerate}
	    \item a = a\_mean\_red, the average of rows of $X$ corresponding to `Red' senators.
	    \item a = a\_mean\_blue, the average of rows of $X$ corresponding to `Blue' senators.
	\end{enumerate}
	
	\sol{\input{voting_data_solution/1.2.tex}
	}
	
	\qitem
	We can compute the variance of scalar projections, $z$, of the the data points (rows of X) along a unit direction $u$ as, 
	\begin{align*}
	    Var(z) = u\tran C u,
	\end{align*}
	where $C = \frac{X\tran X}{n}$.
    Can you express the variance along the first two principal components of PCA, $a_1,a_2$ in terms of eigenvalues of $C$? What is the sum of variance along $a_1$ and $a_2$. Plot the data projected on the plane spanned by $a_1$ and $a_2$.
	
	\sol{\input{voting_data_solution/1.3.tex}
	}
	
	
	\qitem
	Suppose we want to find the bills that are most/least contentious. That is bills for which there is most variability in voting pattern among senators and bills for which the voting is almost unanimous. We can do this in the following two ways:
	\begin{enumerate}
	    \item For each basis vector associated with a bill (That is vector of all zeros with 1 in the entry corresponding to the bill), we can compute score using the basis vector as $a$, and look at variance of the score. This is equivalent to looking at variance of columns of $X$.
	    \item We can look at those bills corresponding to highest and lowest absolute values in the first principal component. This is equivalent to taking inner-products of the first principal component with each of the basis vectors corresponding to the bill and picking the ones with highest/lowest absolute value of inner-products. 
	\end{enumerate}
	Fill in the code in the Jupyter notebook and comment on your observations in the space provided in the notebook.
		
	\sol{\input{voting_data_solution/1.4.tex}  
	}
	\qitem
	Finally we can classify senators as most/least ``extreme'' based on the absolute value of their scores along the first principal component. Comment on your observations in the space provided in the Jupyter notebook.
		
	\sol{\input{voting_data_solution/1.5.tex}  
	}

\end{enumerate}