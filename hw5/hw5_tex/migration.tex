\qns{City migrations as matrix iterations: the transition matrix}

In this exercise we are interested in the evolution of the demographics of cities.
We represent migration as a transition process between cities inside a graph.
This transition process can be represented in a matrix.
We are interested in some properties of this matrix that can be exploited to better understand the system.

The exercise includes questions on the iPython notebook ``city\_migration.ipynb''.

We consider that all the cities are part of a big network. 
The network is represented as a graph which is a set of vertices in $V = \{1,...,n\}$, with an edge joining any pair of vertices in a set $E \subseteq V \times V$.
Every vertex represents a city. 
There exists an edge between the city $i$ and the city $j$ if there is human migration between the city $i$ and the city $j$.

% \todo[inline]{Add a story: \dots Remove the word probability and Markov process.}

% We are given a graph as a set of vertices in $V = \{1,...,n\}$, with an edge joining any pair of vertices in a set $E \subseteq V \times V$. 
% We assume that the graph is undirected (without arrows), meaning that $(i,j)\in E$ implies $(j,i)\in E$.
We define the transition matrix of the graph $T$ by:
\[
T_{ij} = \left\{ 
\begin{array}{ll}
p_{i,j} & \mbox{if } (i,j) \in E, \\
0 & \mbox{otherwise.}
\end{array}
\right.
\]
Where $p_{i,j}$ is the annual percentage of people that leave the city $i$ to go to the city $j$.

Where $p_{i,i}$ represents the annual percentage of people that stay in the city $i$.

Because city population is non-negative, and because it is assumed that each city's population can only change from people entering/exiting from/to other cities (i.e. the total population in the network of cities remains constant over years), we have that:
\begin{align*}
    p_{i,j}\geq 0 &, \quad \quad \forall\ i,j \\
    \sum\limits_{j} p_{i,j} = 1 &, \quad \quad \forall\ i \quad \quad
\end{align*}

We denote $m_{i,t}\geq 0$ the population of the city $i\in V$ at the year $t$. 
We note $\vec{m}_t = (m_{i,t})_{i\in V}$.

\begin{enumerate}
\item Explain why, in this model, the population of every city at the year $t+1$ can be computed as $T\tran \vec{m}_{t}$.

\sol{\item Let $x,y\in\Real{n}$ be two unit-norm vectors, that is, such that $\|x\|_2 = \|y\|_2=1$. Show algebraically that the vectors $x-y$ and $x+y$ are orthogonal. Then, show this graphically by drawing the two vectors on the 2D plane, as well as any other necessary shapes. You may use right angles, circles and straight lines to make your point.}
\item Show that if $\lambda$ is an eigenvalue of $T$, then $|\lambda|\leq 1$.

\textit{Hint:} Consider using $\|\cdot\|_\infty$.

\sol{\item Show that the following inequalities hold for any vector $x \in \Real{n}$:
\[
% \|x\|_\infty \le \|x\|_1 \le n \|x\|_\infty \\
% \|x\|_2 \le \|x\|_1 \le \sqrt{n} \|x\|_2 \\
% \|x\|_\infty \le \|x\|_2 \le\sqrt{n}\|x\|_\infty \\
\frac{1}{\sqrt{n}}\|x\|_2 \leq \|x\|_\infty \leq  \|x\|_2 \leq \|x\|_1 \leq \sqrt{n} \|x\|_2 \le n\|x\|_\infty.
\]

\textit{Hint:} For $\|x\|_1\leq \sqrt{n}\|x\|_2$, how might you express $\|x\|_1$ as the dot product of two vectors? Can you then use the Cauchy-Schwarz inequality to bound this dot product?}
\item Show that $1$ is always an eigenvalue of $T$. State an eigenvector $\vec{v}$ of $T$ associated with the eigenvalue $1$, such that $\|\vec{v}\|_2 = 1$.

\sol{\item $g(x) = f(Ax + b) x$ where $f: \mathbb{R}^n \mapsto \mathbb{R}$ is once differentiable and $A \in \mathbb{R}^{n \times n}$.}
\item Show that the total population over every city is constant over time: e.g. show that $\| T\tran\vec{m}\|_1 = \| \vec{m} \|_1$.

\sol{\item
\[
g(x) = \begin{bmatrix}
x_1^2/x_2 \\
\log(x_3) \sin(x_1/x_3)
\end{bmatrix}
\]}
\item Show that finding all the eigenvectors associated with the eigenvalue $1$ is equivalent to find $N(I-T)$. We do not consider the case of the zero vector.

\textit{Hint:} Consider an eigenvector $v$ of $T$ associated with the eigenvalue $1$ and show that $v\in N(I-T)$. Then consider $u\in N(I-T)$ and show that $u$ is an eigenvector of $T$ associated with the eigenvalue $1$.

\sol{\input{migration_solution/5}}
\item Let's consider an eigenvector $\vec{v}$ of $T$ associated with the eigenvalue $1$. Consider $i=\arg\min\limits_j v_j$ and $\vec{u} = \vec{v}- v_i \mathbf{1}$. Show that $T\vec{u}=\vec{u}$, $\vec{u}\geq 0$ and $u_i = 0$.

\sol{\input{migration_solution/6}}
\item Show that $e_i^\top T^n \vec{u} = 0$.

\sol{\input{migration_solution/7}}
\end{enumerate}

One can show that (the proof can be shown through calculus):
\[e_i^\top T^n \vec{u} = \sum\limits_j \left(\sum\limits_{i_1,\dots,i_n} T_{i,i_1} \left( \prod\limits_{k=1}^{n-1} T_{i_k, i_{k+1}} \right) T_{i_n, j} u_j\right)\]

\begin{enumerate}
\setcounter{enumi}{7}
\item Show that $\forall j, {i_1,\dots,i_n}, \  T_{i,i_1} \left( \prod\limits_{k=1}^{n-1} T_{i_k, i_{k+1}} \right) T_{i_n, j} u_j \geq 0$.

\sol{\input{migration_solution/8}}
\item Conclude that $\forall j, {i_1,\dots,i_n},\  T_{i,i_1} \left( \prod\limits_{k=1}^{n-1} T_{i_k, i_{k+1}} \right) T_{i_n, j} u_j = 0$

\sol{\input{migration_solution/9}}
\end{enumerate}

\textbf{Definition (strongly connected)}: A graph is strongly connected if there is a path between any pair of vertices of the graph.

One can also show that if the graph is strongly connected then (the proof is similar to the one for the exercise on the representation of a graph as an adjacency matrix):
\[\forall j, \exists n, \exists i_1,\dots,i_n,\ T_{i,i_1} \left( \prod\limits_{k=1}^{n-1} T_{i_k, i_{k+1}} \right) T_{i_n, j} \neq 0\]


Now let assume that the graph is strongly connected. We are interested to now all the possible eigenvectors associated with the eigenvalue $1$.

\begin{enumerate}
\setcounter{enumi}{9}
\item Show that $\forall j, u_j = 0$.

\sol{\input{migration_solution/10}}
\item Conclude that $\vec{v} = v_i \mathbf{1}$.

\sol{\input{migration_solution/11}}
\item State what is $N(I-T)$ when the graph is strongly connected.

\sol{\input{migration_solution/12}}
\end{enumerate}

Now let assume that $T$ is symmetric.
\begin{enumerate}
\setcounter{enumi}{12}
\item Explain what is the meaning of a symmetric transition matrix for the city's population.

\sol{\input{migration_solution/13}}
\end{enumerate}


Now let assume that the graph is strongly connected, that $T$ is symmetric and that $-1$ is not an eigenvector of $T$.
\begin{enumerate}
\setcounter{enumi}{13}
\item Show that $\lim \limits_{l\rightarrow \infty} T^l = \frac{\mathbf{11}^\top}{n}$.

\textit{Hint:} use answers of question b., c. and l.

\sol{\input{migration_solution/14}}
\item Conclude that if $T$ is constant over time, if $T$ is symmetric, if the graph of the cities is strongly connected, and if $-1$ is not an eigenvalue of $T$, then the population of every city will be the same after a given time.

\sol{\input{migration_solution/15}}
\item The exercise continues on the iPython notebook ``city\_migration.ipybn''. Please answer the question ask there.
\end{enumerate}

\begin{enumerate}
\item[Bonus.] Show that the previous answer is false if Stanford and Berkeley are part of the cities considered.

\sol{\input{migration_solution/bonus}}
\end{enumerate}


Some of you might have recognize that $p$ can be interpreted as a probability, and that the transition matrix can be interpreted as a stochastic matrix. 
The equation $\mathbf{m}_{t+1} = T^\top \mathbf{m}_{t}$ encodes a Markov process.

If you enjoyed this exercise feel free to take a look at:
\begin{itemize}
    \item For the maths lovers
    \begin{itemize}
        \item The Perron-Frobenius theorem: \url{https://en.wikipedia.org/wiki/Perron–Frobenius_theorem#Applications}
        \item Stochastic matrix on Wikipedia: \url{https://en.wikipedia.org/wiki/Stochastic_matrix}
    \end{itemize}
    \item For the sociologists
    \begin{itemize}
        \item Bonacich, P. (1987) \href{http://www.leonidzhukov.net/hse/2014/socialnetworks/papers/Bonacich-Centrality.pdf}{‘Power and Centrality: A Family of Measures’}, American Journal of Sociology, 92, 1170 – 82.
        \item Freeman, L. C. (1978/79) \href{http://www.leonidzhukov.net/hse/2018/sna/papers/freeman79-centrality.pdf}{‘Centrality in Social Networks: Conceptual Clarification’}, Social Networks, 1, 215 – 39.
    \end{itemize}
\end{itemize}
