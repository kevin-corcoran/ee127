\qns{Traffic flow: the return of the transition matrix}

In this exercise, we use the transition matrix to describe the traffic flow evolution inside a road network.

We are given a graph as a set of vertices in $V = \{1,...,n\}$, with an edge joining any pair of vertices in a set $E \subseteq V \times V$. 
We define the transition matrix of the graph $T$ by:
\[
T_{ij} = \left\{ 
\begin{array}{ll}
p_{i,j} & \mbox{if } (i,j) \in E, \\
0 & \mbox{otherwise.}
\end{array}
\right.
\]
Where $p_{i,j}$ is the probability to go from $i$ to $j$:
\begin{align*}
    p_{i,j}\geq 0 &, \quad \quad \forall\ i,j \\
    \sum\limits_{j} p_{i,j} = 1 &, \quad \quad \forall\ i
\end{align*}

In a graph where vertices represent road segment and edges represent intersections, we denote $f_i\geq 0$ the traffic flow on the road segment $i\in V$ at the time $0$.

\underline{Remark:} In the graph, vertices represent road segment and edges represent intersections. This might be counter-intuitive, as you might think about the road network where road segment are edges or links of the network.
The graph representation where vertices represent road segment of the road network is called line graph of the road network.

We denote $p_{i,j} f_{i}$ the traffic flow on the road segment $i$ that will go to the road segment $j$ at the intersection $(i,j)$ during the time step $\delta t$.
The number of people staying on the road segment is then $p_{i,i} f_i$.

% THE GRAPH CAN BE DIRECTED, THEREFORE $T$ MIGHT BE NOT SYMMETRIC.

\begin{enumerate}
\item Explain why if there is no input flow then the flow at the time $\delta t$ will be $\vec{f_{\delta t}} = T^\top \vec{f_{0}}$.

\sol{\item Let $x,y\in\Real{n}$ be two unit-norm vectors, that is, such that $\|x\|_2 = \|y\|_2=1$. Show algebraically that the vectors $x-y$ and $x+y$ are orthogonal. Then, show this graphically by drawing the two vectors on the 2D plane, as well as any other necessary shapes. You may use right angles, circles and straight lines to make your point.}
\end{enumerate}

Now let's assume that at each time step some people enter the road network and some people exit the road network.
We define $D$ as the equilibrium demand matrix of the road network.
$D_{ij}$ denotes the number of people entering $i$ at each time step intending to exit at $j$. 
This means at the equilibrium, $D_{ij}$ people enter road $i$ in order to travel to road $j$ and $D_{ij}$ leave road $j$ after having travelled from road $i$ at every time step (time step of $\delta t$). 

The amount of people entering the network is equal to the amount of people exiting the network at every time step; in this sense, $D$ is the equilibrium demand matrix.


We are interested in the finding the flow at time $t+\delta t$ as a function of the demand $D$ and of the flow at time $t$.
We assume that the equilibrium is reached. 

\begin{enumerate}
\setcounter{enumi}{1}
\item Show that $\vec{f_{t+\delta t}} = T^\top \vec{f_t} + (D - D^\top) \mathbf{1}$.

\sol{\item Show that the following inequalities hold for any vector $x \in \Real{n}$:
\[
% \|x\|_\infty \le \|x\|_1 \le n \|x\|_\infty \\
% \|x\|_2 \le \|x\|_1 \le \sqrt{n} \|x\|_2 \\
% \|x\|_\infty \le \|x\|_2 \le\sqrt{n}\|x\|_\infty \\
\frac{1}{\sqrt{n}}\|x\|_2 \leq \|x\|_\infty \leq  \|x\|_2 \leq \|x\|_1 \leq \sqrt{n} \|x\|_2 \le n\|x\|_\infty.
\]

\textit{Hint:} For $\|x\|_1\leq \sqrt{n}\|x\|_2$, how might you express $\|x\|_1$ as the dot product of two vectors? Can you then use the Cauchy-Schwarz inequality to bound this dot product?}
\end{enumerate}

Now we are interested in finding the equilibrium flow: $f_{eq}$ such that $f_{eq} = T^\top f_{eq} + (D - D^\top) \mathbf{1}$.

\begin{enumerate}
\setcounter{enumi}{2}
\item Show that finding $\vec{f_{eq}}$ is a solution of a linear equation $A\vec{x}=\vec{b}$. State $A$ and $\vec{b}$.

\sol{\item $g(x) = f(Ax + b) x$ where $f: \mathbb{R}^n \mapsto \mathbb{R}$ is once differentiable and $A \in \mathbb{R}^{n \times n}$.}
\item Show that there is a solution if and only if $\vec{b}\perp N(A^\top)$.

\sol{\item
\[
g(x) = \begin{bmatrix}
x_1^2/x_2 \\
\log(x_3) \sin(x_1/x_3)
\end{bmatrix}
\]}
\item Assuming that the graph is strongly connected, we know that $N(I-T)=Span(\textbf{1})$ (c.f. first exercise on the transition matrix). Then, show that there exists a equilibrium flow if $\mathbf{1}^\top \vec{b} = 0$.

\sol{\input{network_flow_solution/5}}
\item Show that $\mathbf{1}^\top \vec{b} = 0$.

\sol{\input{network_flow_solution/6}}
\item Explain why $\mathbf{1}^\top \vec{b} = 0$ encodes the flow conservation inside the network.

\sol{\input{network_flow_solution/7}}
\item Let assume that, if the network is strongly connected, $\exists \vec{f^*} > 0$ such that $N(A)=Span(\vec{f^*})$ (Perron-Frobenius theorem). Show that, if the network is strongly connected, there exists a equilibrium flow $\vec{f_{eq}}$ such that $\vec{f_{eq}}\geq 0$.

\sol{\input{network_flow_solution/8}}
\end{enumerate}
\underline{Remark}: this was the technique used by Google in the beginning of their search engine to compute the page rank. $i\in V$ is a website, $e\in E$ is an hyperlink between two website, and $p_{i,j}$ is the probability to click on the hyperlink on the website $i\in V$ that will lead you to the website $j\in V$. $\vec{f_{eq}}$ is then the ranking of the websites. 

Now, let assume that the network contains no cycle.
\begin{enumerate}
\setcounter{enumi}{8}
\item Show that the network is not strongly connected.

\sol{\input{network_flow_solution/9}}
\end{enumerate}

One can show that if the network contains no cycle then $(T^{\top})^n = 0$
\begin{enumerate}
\setcounter{enumi}{9}
\item Show that $1$ is not an eigenvalue of $T$.

\sol{\input{network_flow_solution/10}}
\item Show that $(I-T^\top) (\sum\limits_{k=0}^{n-1} (T^\top)^k) = I$.

\sol{\input{network_flow_solution/11}}
\item Show that in this case $\vec{f_{eq}}$ exists and $\vec{f_{eq}} = (\sum\limits_{k=0}^{n-1} (T^\top)^k)(D - D^\top) \mathbf{1}$.

\sol{\input{network_flow_solution/12}}
\end{enumerate}
Remark that we have not shown that $\vec{f_{eq}}\geq 0$. This would have required a little bit more questions.

\begin{enumerate}
\setcounter{enumi}{13}
\item[Bonus.] With regard to this exercise, explain why it is better to use neural networks without understanding the underlying math.

\sol{\input{network_flow_solution/bonus}}
\end{enumerate}

If you enjoyed this exercise, feel free to take a look at:
\begin{itemize}
    \item A Google-like model of road network dynamics and its application to regulation and control, E.Crisostomi, S.Kirkland, R.Shorten
\end{itemize}