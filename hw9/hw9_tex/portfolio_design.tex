%!TEX root = ./hw3.tex
\qns{A Portfolio Design Problem}\rm
\label{exer:portfolio_socp}
The returns on $n = 4$ assets are described by a Gaussian (normal) random vector $r \in \mathbb{R}^n$ , having the following expected value $\hat{r}$ and covariance matrix $\Sigma$: 
\[
\hat{r} = 
\begin{bmatrix}
    0.12  \\
    0.10  \\
    0.07 \\
    0.03 
    
\end{bmatrix}, \; \Sigma = 
\begin{bmatrix}
    0.0064 & 0.0008 & -0.0011  & 0 \\
    0.0008 & 0.0025 & 0  & 0 \\
    -0.0011 & 0 & 0.0004  & 0 \\
    0 & 0 & 0  & 0
    
\end{bmatrix}
.
\]


The last (fourth) asset corresponds to a risk-free investment. An investor
wants to design a portfolio mix with weights $x \in \mathbb{R}^n$ (each weight $x_i$
is non-negative, and the sum of the weights is one) so as to obtain the best possible expected return $\hat{r}^Tx$, while guaranteeing that: \\ 

(i) No single asset weighs more than $40\%$. \\ 
(ii) The risk-free assets should not weigh more than $20\%$. \\ 
(iii) No asset should weigh less than $5\%$. \\ 
(iv) The probability of experiencing a return lower than $q = -1\%$ should be no larger than $\epsilon = 10^{-4}$.
\begin{enumerate}

\item {[Short justification, 1-2 lines]} For a scalar standard normal $y \sim \mathcal{N}(0, 1)$, show that the constraint

\begin{equation*}
    \textbf{Pr}(y \leq q) \leq \epsilon
\end{equation*}

can be written as

\begin{equation*}
    q \leq \Phi^{-1}(\epsilon),
\end{equation*}

where $\Phi(y)$ is the CDF of $y$, and $\Phi^{-1}$ is the inverse CDF. In particular, justify taking any inverse across inequality.

\newline
\sol{\item Let $x,y\in\Real{n}$ be two unit-norm vectors, that is, such that $\|x\|_2 = \|y\|_2=1$. Show algebraically that the vectors $x-y$ and $x+y$ are orthogonal. Then, show this graphically by drawing the two vectors on the 2D plane, as well as any other necessary shapes. You may use right angles, circles and straight lines to make your point.}
\newline

\item {[Short justification, 1-2 lines]} Given a multivariable Gaussian $r \sim \mathcal{N}(\hat{r}, \Sigma),r \in \mathcal{R}^n$, its projection onto any $x \in \mathcal{R}^n$ is also a scalar Gaussian. That is,

\begin{equation*}
    r^Tx \sim \mathcal{N}(\hat{r}^Tx, x^T \Sigma x).
\end{equation*}

Show that the fourth constraint, also known as the chance constraint, can be written as 

\begin{equation*}
\Phi^{-1}(10^{-4})||\Sigma^{1/2}x||_2 \geq -\hat{r}^Tx - 0.01.
\end{equation*}
Note that $\Phi^{-1}(10^{-4})$ is negative so this constraint can be converted to one in standard SOCP format as 
\begin{equation*}
||\Sigma^{1/2}x||_2 \leq \frac{1}{\Phi^{-1}(10^{-4})}(-\hat{r}^Tx - 0.01).
\end{equation*}

\emph{Hint: Write $r\tran x$ as a standard Normal.}

\newline
\sol{\item Show that the following inequalities hold for any vector $x \in \Real{n}$:
\[
% \|x\|_\infty \le \|x\|_1 \le n \|x\|_\infty \\
% \|x\|_2 \le \|x\|_1 \le \sqrt{n} \|x\|_2 \\
% \|x\|_\infty \le \|x\|_2 \le\sqrt{n}\|x\|_\infty \\
\frac{1}{\sqrt{n}}\|x\|_2 \leq \|x\|_\infty \leq  \|x\|_2 \leq \|x\|_1 \leq \sqrt{n} \|x\|_2 \le n\|x\|_\infty.
\]

\textit{Hint:} For $\|x\|_1\leq \sqrt{n}\|x\|_2$, how might you express $\|x\|_1$ as the dot product of two vectors? Can you then use the Cauchy-Schwarz inequality to bound this dot product?}
\newline

\item {[Optimization Problem Formulation]} Formulate the portfolio optimization problem as a SOCP.
Solve the problem using CVXPY in the Jupyter notebook. What is the maximal achievable expected return, under the above constraints? \\


% \textbf{Hint}: Constraint (iv) is known as a "chance constraint." 
% \[
% a^Tx \sim N(\hat{a}, \Sigma) \implies a^Tx - b \sim N(\hat{a}x - b, x^Tx\Sigma x).
% \]
% We then have:
% \[\Pr(a^Tx \leq b) \geq \eta \iff b - \hat{a}^Tx \geq \Phi^{-1}(\eta)||\Sigma^{1/2}x||_2
% \]

\newline
\sol{\item $g(x) = f(Ax + b) x$ where $f: \mathbb{R}^n \mapsto \mathbb{R}$ is once differentiable and $A \in \mathbb{R}^{n \times n}$.}
\newline

\item Solve the problem for a large number of values of $\epsilon$ between $10^{-6}$ and $10^{-1}$ and plot the optimal values of $\hat{r}^Tx$ versus $\epsilon$.   Also make an area plot of the optimal portfolios $x$ versus $\epsilon$. What do you observe as the risk tolerance $\epsilon$ decreases? Read  \href{https://en.wikipedia.org/wiki/Area_chart}{\color{blue} here} for what an area plot is. 

\newline
\sol{\item
\[
g(x) = \begin{bmatrix}
x_1^2/x_2 \\
\log(x_3) \sin(x_1/x_3)
\end{bmatrix}
\]}
\newline

\item \textit{Monte Carlo simulation.} Let $x$ be the optimal portfolio found in part 1, with $\epsilon = 10^{-4}$ . This portfolio maximizes the expected return, subject to the probability of a loss being no more than 1 $\%$. Generate 10000 samples of $r$, and plot a histogram of the returns. Find the empirical mean of the return, and calculate the percentage of samples for which a loss occurs. For more on Monte Carlo, see \href{https://en.wikipedia.org/wiki/Monte_Carlo_method}{\color{blue} here}.

\newline
\sol{\input{portfolio_design_sol/5.tex}}
\newline

\end{enumerate}

