\qns{Gram-Schmidt}
\newline
Any set of $n$ linearly independent vectors in $\mathbb R^n$ could be used as a basis for $\mathbb R^n$. However, certain bases could be more suitable for certain operations than others. For example, an orthonormal basis could facilitate solving linear equations.

\begin{enumerate}

\item Given a matrix $A \in \mathbb R^{n \times n}$, it could be represented as a multiplication of two matrices
\[ A = Q R, \]
where $Q$ is a unitary matrix (its columns form an orthonormal basis for $\mathbb R^n$) and $R$ is an upper-triangular matrix. For the matrix $A$, describe how Gram-Schmidt process could be used to find the $Q$ and $R$ matrices, and apply this to 
\[ A = 
\begin{bmatrix} 3 & -3 & 1\\
4 & -4 & -7 \\
0 & 3 & 3
\end{bmatrix}
\] 
to find a unitary matrix $Q$ and an upper-triangular matrix $R$.

\sol{\item Let $x,y\in\Real{n}$ be two unit-norm vectors, that is, such that $\|x\|_2 = \|y\|_2=1$. Show algebraically that the vectors $x-y$ and $x+y$ are orthogonal. Then, show this graphically by drawing the two vectors on the 2D plane, as well as any other necessary shapes. You may use right angles, circles and straight lines to make your point.}

\item Given an invertible matrix $A \in \mathbb R^{n \times n}$ and an observation vector $b \in \mathbb R^n$, the solution to the equality
 \[ A x = b \]
is given as $x = A^{-1}b$. For the matrix $A = QR$ from part (a), assume that we want to solve
\[ A x = \begin{bmatrix}
8 \\ -6 \\ 3
\end{bmatrix}. \]
By using the fact that $Q$ is a unitary matrix, find $\overline b$ such that
\[ R x = \overline b. \]

Then, given the upper-triangular matrix $R$ and $\overline b$ in part (c), find the elements of $x$ \underline{sequentially}.

\sol{\item Show that the following inequalities hold for any vector $x \in \Real{n}$:
\[
% \|x\|_\infty \le \|x\|_1 \le n \|x\|_\infty \\
% \|x\|_2 \le \|x\|_1 \le \sqrt{n} \|x\|_2 \\
% \|x\|_\infty \le \|x\|_2 \le\sqrt{n}\|x\|_\infty \\
\frac{1}{\sqrt{n}}\|x\|_2 \leq \|x\|_\infty \leq  \|x\|_2 \leq \|x\|_1 \leq \sqrt{n} \|x\|_2 \le n\|x\|_\infty.
\]

\textit{Hint:} For $\|x\|_1\leq \sqrt{n}\|x\|_2$, how might you express $\|x\|_1$ as the dot product of two vectors? Can you then use the Cauchy-Schwarz inequality to bound this dot product?}

\item Describe how your solution in the previous problem is akin to Gaussian elimination in solving a system of linear equations.

\sol{\item $g(x) = f(Ax + b) x$ where $f: \mathbb{R}^n \mapsto \mathbb{R}$ is once differentiable and $A \in \mathbb{R}^{n \times n}$.}

\item Given an invertible matrix $B \in \mathbb R^{n \times n}$ and an observation vector $c \in \mathbb R^n$, find the computational cost of finding the solution $z$ to the equation $Bz = c$ by using the $QR$ decomposition of $B$. Assume that $Q$ and $R$ matrices are available, and adding, multiplying, and dividing scalars take one unit of ``computation". 

As an example, computing the inner product $a\tran b$ is said to be $O(n)$, since we have $n$ scalar multiplications total -- one for each $a_i b_i$. Similarly, matrix vector multiplication is $O(n^2)$, since matrix vector multiplication can be viewed as computing $n$ inner products. The computational cost for inverting a matrix in $\mathbb R^n$ is $O(n^3)$, and consequently, the cost grows rapidly as the set of equations grows in size. This is why the expression $A^{-1}b$ is usually not computed by directly inverting the matrix $A$. Instead, the $QR$ decomposition of $A$ is exploited to decrease the computational cost.

\sol{\item
\[
g(x) = \begin{bmatrix}
x_1^2/x_2 \\
\log(x_3) \sin(x_1/x_3)
\end{bmatrix}
\]}

\end{enumerate}