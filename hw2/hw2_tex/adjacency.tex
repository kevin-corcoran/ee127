\qns{Representation of a graph as a matrix: the adjacency matrix}
% \todo[inline]{Better explain the notations, refer to the definition of the notations.}

In this exercise, we are interpreting a graph as a matrix. 
Then, we show that finding a path between two nodes can be done with a matrix multiplication.  
This can be useful to quickly compute several shortest paths, or to re-compute shortest paths when the weights on the links of the graph only change slightly.

We are given a graph as a set of vertices $V = \{1,...,n\}$, with edges $E \subseteq V \times V$. We assume that the graph is undirected (without arrows), meaning that $(i,j)\in E$ implies $(j,i)\in E$.

We define the adjacency matrix of the graph $A$ by:
\[
A_{ij} = \left\{ 
\begin{array}{ll}
1 & \mbox{if } (i,j) \in E, \\
0 & \mbox{otherwise.}
\end{array}
\right.
\]

\begin{figure}[h]
\centering
\includegraphics[width=.4\textwidth]{figures/UndirectedGraph.png}
\caption[Graph example]{Example of an undirected graph.}
\label{fig:symm-graph}
\end{figure}

\begin{enumerate}
\item Form the adjacency matrix for the graph shown in Figure~\ref{fig:symm-graph}.

\sol{\item Let $x,y\in\Real{n}$ be two unit-norm vectors, that is, such that $\|x\|_2 = \|y\|_2=1$. Show algebraically that the vectors $x-y$ and $x+y$ are orthogonal. Then, show this graphically by drawing the two vectors on the 2D plane, as well as any other necessary shapes. You may use right angles, circles and straight lines to make your point.}
\item Turning to a generic graph, show that the adjacency matrix $A$ is symmetric.

\sol{\item Show that the following inequalities hold for any vector $x \in \Real{n}$:
\[
% \|x\|_\infty \le \|x\|_1 \le n \|x\|_\infty \\
% \|x\|_2 \le \|x\|_1 \le \sqrt{n} \|x\|_2 \\
% \|x\|_\infty \le \|x\|_2 \le\sqrt{n}\|x\|_\infty \\
\frac{1}{\sqrt{n}}\|x\|_2 \leq \|x\|_\infty \leq  \|x\|_2 \leq \|x\|_1 \leq \sqrt{n} \|x\|_2 \le n\|x\|_\infty.
\]

\textit{Hint:} For $\|x\|_1\leq \sqrt{n}\|x\|_2$, how might you express $\|x\|_1$ as the dot product of two vectors? Can you then use the Cauchy-Schwarz inequality to bound this dot product?}
\end{enumerate}

Now let's show that if there is a path between $i\in V$ and $j\in V$, then there exists $n\in \mathbb{N}$ such that $e_i^\top A^n e_j \neq 0$. We define $e_i$ as the vector of length $n$ where the $i$th entry is $1$ and all other entries are $0$. 

Let's proceed by induction. Define the following property:
\begin{align}
    P(n) : \text{ there is a path of length } n \text{ between } i \text{ and } j \implies e_i^\top A^n e_j \neq 0
\end{align}


And show that $P(1)$ is true and that $\forall n>1,\ P(n-1)\implies P(n)$. This will show that $\forall n\geq 1$, $P(n)$ is true.

% Formally the length $n$ of a path $p$ is the number of edges that it contains.

To get some intuition about the exercise that you will now solve, it is strongly recommended to run the iPython notebook ``Adjacency matrix.ipynb''.
You do \textbf{NOT} need to submit the iPython notebook to gradescope.

Let assume that there is a path -- denoted $p$ -- between $i$ and $j$. We denote the length of the path $p$ as $n$.

\begin{enumerate}
\setcounter{enumi}{2}
\item Show that if $n=1$ -- meaning that there is a path of length $1$ between $i$ and $j$ -- then $e_i^\top A^n e_j \neq 0$. State what is the value of $e_i^\top A^n e_j$.

\sol{\item $g(x) = f(Ax + b) x$ where $f: \mathbb{R}^n \mapsto \mathbb{R}$ is once differentiable and $A \in \mathbb{R}^{n \times n}$.}
\end{enumerate}

Now let's assume that the property is true for $n-1$, with $n>1$. Let show that it is true for $n$.

\begin{enumerate}
\setcounter{enumi}{3}
\item Show that $e_i^\top A^n e_j = \sum\limits_k (e_i^\top A e_k) (e_k^\top A^{n-1} e_j)$.
This equation means that any path of length $n$ between $i$ and $j$ is the combination of a link between $i$ and a vertices $k$ and a path of length $n-1$ from $k$ to $j$.

\sol{\item
\[
g(x) = \begin{bmatrix}
x_1^2/x_2 \\
\log(x_3) \sin(x_1/x_3)
\end{bmatrix}
\]}
\item Show that $\forall k, (e_i^\top A e_k) (e_k^\top A^{n-1} e_j)\geq 0$.

\sol{\input{adjacency_solutions/5}}
\end{enumerate}

Let's denote $l$ the first node achieved in the path $p$ from $i$.

\begin{enumerate}
\setcounter{enumi}{5}
\item Explain why $e_l^\top A^{n-1} e_j \neq 0$.

\sol{\input{adjacency_solutions/6}}
\item State the value of $e_i^\top A e_l$.

\sol{\input{adjacency_solutions/7}}
\item Conclude that $e_i^\top A^n e_j \neq 0$.

\sol{\input{adjacency_solutions/8}}
\item Have you shown that there is a path between $i$ and $j$ if and only if $e_i^\top A^n e_j\neq 0$ ?

\sol{\input{adjacency_solutions/9}}

If you enjoyed this exercise, feel free to read:
\begin{itemize}
    \item Algebraic Graph Theory, C.Godsil, G.Royle
    \item Algebraic Graph Algorithms, P.Sankowski
\end{itemize}

\end{enumerate}