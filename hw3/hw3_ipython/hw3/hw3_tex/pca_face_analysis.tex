\qns{PCA and face analysis}

We have seen how to represent images as matrices. Similarly, we may represent an image as a vector. In this exercise we will explore analyzing a dataset of images represented as vectors. 

We will analyze a dataset of human faces to see if we can learn anything meaningful from it. One way to analyze a dataset is to look for directions of maximal variation. That is, in the space of the data we look for the directions in which the data samples change value the most.

Let $\vec{x}_i \in \mathbb{R}^m, i = 1, \hdots, n$, be the given data points to analyze, denote $\hat{\vec{x}} =  \frac{1}{n}\sum_{i =1}^{n} \vec{x}_i $, the mean of the data points with $ \tilde{\vec{x}_i} = \vec{x}_i - \hat{\vec{x}}$, the centered datapoint. Let 
\begin{equation*}
\Tilde{X} = 
\begin{bmatrix}
    \leftarrow \tilde{\vec{x}}_1^{\top} \rightarrow \\
    \leftarrow \tilde{\vec{x}}_2^{\top} \rightarrow \\
    \vdots       \\
    \leftarrow \tilde{\vec{x}}_n^{\top} \rightarrow
\end{bmatrix}
\in \mathbb{R}^{n \times m}
\end{equation*}
be the matrix such that the $i$th row is $\Tilde{\vec{x}}_i^{\top}$.
\\
Then the components of the centered data along a direction (by direction, we mean unit norm vector) $\vec{z}$ are given by:
\begin{equation*}
    \alpha_{i} = \Tilde{\vec{x}}_i^{\top}\vec{z}, \hspace{5pt} i = 1, \hdots, n
\end{equation*}
\\
The mean square variation of the data along direction $\vec{z}$ is given by 
\begin{equation*}
    \frac{1}{n} \sum_{i=1}^{n} \alpha_{i}^2 = \frac{1}{n}\sum_{i=1}^{n} \vec{z}^{\top}\tilde{\vec{x}}_{i}\tilde{\vec{x}}_{i}^{\top}\vec{z} = \frac{1}{n}\vec{z}^{\top}\tilde{X}^{\top}\tilde{X}\vec{z}
\end{equation*}
\\
The direction $\vec{z}$ along which the data has the largest variation can thus be found as the solution to the following optimization problem:
\begin{equation*}
    \underset{\begin{subarray}{c}
  \vec{z} \in \mathbb{R}^{m} \\
  \text{s.t. } ||\vec{z}||_2 = 1
  \end{subarray}}{\text{max }} \vec{z}^{\top}\tilde{X}^{\top}\tilde{X}\vec{z}
\end{equation*}

\begin{enumerate}
\item
   Show that the direction of maximal variation is the direction of the eigenvector corresponding to the largest eigenvalue of the matrix $\tilde{X}^{\top}\tilde{X}$.
    \newline

\sol{\input{pca_face_analysis_solution/1.1.tex}}

\item
   How is this related to the singular value decomposition of $\tilde{X}$?
    \newline
    
\sol{\input{pca_face_analysis_solution/1.2.tex}}

 \item
    Suppose we wish to find the direction with the second-largest variance, and that $v_1$ is the  direction of maximal variation. We can construct the following transformation to the data points:
    $$\tilde{\vec{x}}_{i}^{(1)} = \tilde{\vec{x}}_i - \vec{v}_1(\vec{v}_1^{\top}\tilde{\vec{x}}_{i}):i = 1,\hdots, n.$$
    This transformation simply removes the components along the direction of maximal variation from the data points.
    \\
    We can then define the matrix 
    \begin{equation*}
    \tilde{X}^{(1)} = 
    \begin{bmatrix}
    \leftarrow \tilde{\vec{x}}^{(1)\top}_1 \rightarrow \\
    \leftarrow \tilde{\vec{x}}^{(1)\top}_2 \rightarrow \\
    \vdots       \\
    \leftarrow \tilde{\vec{x}}^{(1)\top}_n \rightarrow
    \end{bmatrix}
    \in \mathbb{R}^{n \times m}
    \end{equation*}
    \\
    The direction of second largest variation can then be found by solving the optimization problem:
    \begin{equation*}
        \underset{\begin{subarray}{c}
        \vec{z} \in \mathbb{R}^{m} \\
         \text{s.t. } ||\vec{z}||_2 = 1
    \end{subarray}}{\text{max }} \vec{z}\tilde{X}^{(1)\top}\tilde{X}^{(1)}\vec{z}
\end{equation*}
    \\
    Show that the direction with the second largest variation is the direction of the eigenvector corresponding to the second largest eigenvalue of the matrix $\tilde{X}^{\top}\tilde{X}$ (Solutions that show this by doing the algebra are preffered).
    
    We can repeat the procedure to find more directions of variation. % TODO: Say something about why called principal components. 
    We will use this procedure to find the directions of maximal variation in our dataset of images. 
    \newline
    
    \sol{\input{pca_face_analysis_solution/1.3.tex}}
    
\item
     Loading images: In the IPython notebook for this assignment, complete the function loadImage. The function cv2.imread(filename), takes in a file name and returns the image at the file location as a matrix. Complete the function so that the matrix becomes a vector. (Hint: You may use numpy's flatten)
     
     Remember to \texttt{conda activate ee127} before starting! 
    \newline

\sol{\input{pca_face_analysis_solution/1.4.tex}}

 \item
    Finding directions of largest variation: We will not make you write the code to solve the optimization problem. Rather, this time we will make use of \texttt{sklearn.decomposition.PCA} to obtain these directions. Use the \texttt{PCA} object's \texttt{fit} method to run SVD, then complete the IPython code to obtain the mean of $X$ and the first 15 directions of largest variation in the dataset using the \texttt{PCA} object. What do you expect these directions to look like intuitively?
    \newline
    
\sol{\input{pca_face_analysis_solution/1.5.tex}}

\item
    Visualizing directions of maximal variation: We will now try to see what each of the directions looks like. To do this, we will take an image and vary it along these directions, one at a time, visualizing what it does to the image. An image has been pre-selected for you. Complete the line of code to reshape the image from a vector representation to a matrix. (Hint: You may use numpy's reshape)
    \newline
    
\sol{\input{pca_face_analysis_solution/1.6.tex}}
\item
    Each of the sliders represents a direction of variation with the first slider being the direction with the largest variation, the next being the second largest and so on. Vary the sliders one at a time, while keeping the other sliders fixed, observing what happens to the pre-selected image. What about the image does the direction of maximal variation change? Does this seem reasonable to you, and why? Take two screenshots of you varying the direction of maximal variation (first slider) only. One screenshot with the slider all the way to the left and another screenshot with the slider all the way to the right, with other sliders held fixed.  You may also experiment with creating new images by varying the different sliders. Have fun creating new faces!
\sol{\input{pca_face_analysis_solution/1.7.tex}}
\end{enumerate}