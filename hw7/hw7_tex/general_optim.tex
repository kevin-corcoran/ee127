\qns{General optimization}

In this exercise, we test your understanding of the general framework of optimization and its language. We consider an optimization problem in standard form: 
\[
p^* = \min_{x \in \Real{n}} \: f_0(x) ~:~ f_i(x) \le 0, \;\; i=1,\ldots,m.
\]
In the following we denote by ${\cal X}$ the feasible set. For the following statements, provide a proof or counter-example.

	
\begin{enumerate}
\item 
{[Short justification, 1--2 lines.]} Any optimization problem can be expressed as one with a linear objective. 

\sol{\item $g(x_1,x_2) = \dfrac{x_1^2}{4} + \dfrac{x_2^2}{9}$


}
\item 
{[Short justification, 1--2 lines.]} Any optimization problem can be expressed as one without any constraints. 

\sol{\item Show that the following inequalities hold for any vector $x \in \Real{n}$:
\[
% \|x\|_\infty \le \|x\|_1 \le n \|x\|_\infty \\
% \|x\|_2 \le \|x\|_1 \le \sqrt{n} \|x\|_2 \\
% \|x\|_\infty \le \|x\|_2 \le\sqrt{n}\|x\|_\infty \\
\frac{1}{\sqrt{n}}\|x\|_2 \leq \|x\|_\infty \leq  \|x\|_2 \leq \|x\|_1 \leq \sqrt{n} \|x\|_2 \le n\|x\|_\infty.
\]

\textit{Hint:} For $\|x\|_1\leq \sqrt{n}\|x\|_2$, how might you express $\|x\|_1$ as the dot product of two vectors? Can you then use the Cauchy-Schwarz inequality to bound this dot product?}
\item 
{[Short justification, 1--2 lines.]} Any optimization problem can be recast as a linear program, provided one allows for an infinite number of constraints. 

\sol{\item $g(x) = \sin(x_1^2) \log (x_3 - x_2)$ where $x_i$ are scalars and $x_3 - x_2 > 0$.}
\item 
{[Short justification, 1--2 lines.]} If one inequality is strict at the optimum, then we can remove it from the original problem and obtain the same solution. 

\sol{\item
\[
g(x) = \begin{bmatrix}
x_1^2/x_2 \\
\log(x_3) \sin(x_1/x_3)
\end{bmatrix}
\]}
\item 
{[A few lines of justification.]} If the problem involves the minimization over more than one variable, say $y$ and $x$, then we can exchange the minimization order without altering the optimal value:
		\[
		\min_x\min_y \: F_0(x,y) = \min_y\min_x \: F_0(x,y)
		\]


\sol{\input{general_optim_solutions/5}}
\item 
{[A few lines of justification.]} If the problem involves the minimization of an objective function of the form
		\[
		f_0(x) = \max_y \: F_0(x,y),
		\]
        in which case
        \[ p^* = \min_{x \in \mathcal{X}} \max_y F_0(x, y)\]
		then $p^* \ge d^*$, where
		\[
		d^* := \max_y \: \min_{x \in {\cal X}} \: F_0(x,y),
		\]
		where ${\cal X}$ is the feasible set of the original problem. {\em Hint:} consider the function $y \rightarrow \min_{x'} \: F_0(x',y)$ and a similar function of $x$.

\sol{\input{general_optim_solutions/6}}
\end{enumerate}