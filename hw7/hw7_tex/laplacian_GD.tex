\qns{Gradient Descent $\times$ The Laplacian}

We are given an undirected simple graph $G=(V,E)$ where $V=\{1,\ldots,n\}$ is the set of vertices and $E\subseteq V\times V$ is the set of edges joining vertices. Since $G$ is undirected, for $i,j\in V$, we have $(i,j)\in E\Leftrightarrow (j,i)\in E$. Let $d_i$ be the degree of vertex $i\in V$, i.e. the number of edges incident to the vertex. We define the Laplacian matrix $L\in\R{n\times n}$ as
\begin{equation*}
    L_{ij}=
    \begin{cases}
    d_i & i=j\\
    -1 & (i,j)\in E\\
    0 & \text{otherwise}
    \end{cases}.
\end{equation*}
Equivalently, $L=D-A$, where $D$ is the diagonal matrix defined by $D_{ii}=d_i$, and $A$ is the adjacency matrix.

\begin{figure}[h!]
    \centering
    \includegraphics{"figures/UndirectedGraph"}
    \caption{An undirected simple graph.}
    \label{fig:graph}
\end{figure}

\begin{enumerate}
    \item {[Write the matrix.]} Form the Laplacian for the graph shown in Figure~\ref{fig:graph}.
    
    \sol{\item Let $x,y\in\Real{n}$ be two unit-norm vectors, that is, such that $\|x\|_2 = \|y\|_2=1$. Show algebraically that the vectors $x-y$ and $x+y$ are orthogonal. Then, show this graphically by drawing the two vectors on the 2D plane, as well as any other necessary shapes. You may use right angles, circles and straight lines to make your point.}
    \item {[A couple lines of justification.]}
    Show that in general, the Laplacian $L$ for any undirected $G=(V,E)$ is symmetric.
    
    \sol{\item Show that the following inequalities hold for any vector $x \in \Real{n}$:
\[
% \|x\|_\infty \le \|x\|_1 \le n \|x\|_\infty \\
% \|x\|_2 \le \|x\|_1 \le \sqrt{n} \|x\|_2 \\
% \|x\|_\infty \le \|x\|_2 \le\sqrt{n}\|x\|_\infty \\
\frac{1}{\sqrt{n}}\|x\|_2 \leq \|x\|_\infty \leq  \|x\|_2 \leq \|x\|_1 \leq \sqrt{n} \|x\|_2 \le n\|x\|_\infty.
\]

\textit{Hint:} For $\|x\|_1\leq \sqrt{n}\|x\|_2$, how might you express $\|x\|_1$ as the dot product of two vectors? Can you then use the Cauchy-Schwarz inequality to bound this dot product?}
    \item {[Algebraic justification.]}
    Show that for any $x\in \R{n}$,
    \begin{equation*}
        x\tran L x=\dfrac{1}{2}\sum_{(i,j)\in E}(x_i-x_j)^2.
    \end{equation*}
    Deduce that $L\succeq 0$.
    
    \sol{\item $g(x) = f(Ax + b) x$ where $f: \mathbb{R}^n \mapsto \mathbb{R}$ is once differentiable and $A \in \mathbb{R}^{n \times n}$.}
    \item {[Short justification and eigenvector.]}
    Show that zero is always an eigenvalue of $L$. Exhibit a corresponding eigenvector.
    
    \sol{\item
\[
g(x) = \begin{bmatrix}
x_1^2/x_2 \\
\log(x_3) \sin(x_1/x_3)
\end{bmatrix}
\]}
    \item {[Algebraic justification.]}
    Consider the problem of assigning weights $x\in\R{n}$ to each vertex $i\in V$ such that the sum of the weights is close to one and each pair of adjacent vertices has similar weights. That is, we want the solution to the optimization problem
    \begin{equation*}
        x^*=\argmin_{x\in\R{n}} \sum_{(i,j)\in E}(x_i-x_j)^2+2\lambda\left(\sum_{i\in V}x_i-1\right)^2
    \end{equation*}
    where $\lambda\in\R{}$ is a constant. Here, the first term inside the $\argmin$ penalizes adjacent vertices with highly differing weights, while the second term forces the sum of the weights to be close to one. Show that this optimization problem is equivalent to
    \begin{equation*}
        x^*=\argmin_{x\in\R{n}} \dfrac{1}{2}x\tran (L+\lambda \vec{1}\vec{1}\tran)x-\lambda \vec{1}\tran x
    \end{equation*}
    where $\vec{1}$ is the all-ones vector in $\R{n}$.
    
    \sol{\input{laplacian_GD_solutions/5.tex}}
    \item {[Algebraic expression and short justification.]}
    What is the optimal $x^*$?
    
    \sol{\input{laplacian_GD_solutions/6.tex}}
    \item {[Algebraic justification.]}
    Suppose we use gradient descent with step size $\eta>0$ to find the optimal $x^*$. Write the gradient descent step, i.e. express $x_{k+1}$, the $(k+1)$th step of gradient descent, in terms of $x_k$, $L$, $\eta$, and $\lambda$.
    
    \sol{\input{laplacian_GD_solutions/7.tex}}
    \item {[Algebraic justification.]}
    Show that $x_{k+1}-x^*=(I-\eta(L+\lambda\vec{1}\vec{1}\tran))(x_k-x^*)$.
    
    \sol{\input{laplacian_GD_solutions/8.tex}}
    \item {[Algebraic justification.]}
    Deduce that $\|x_k-x^*\|_2\leq \rho^k\|x_0-x^*\|_2$ for some $\rho\in\R{}$, where $x_0$ is the starting point of the gradient descent. Express $\rho$ in terms of $\eta$, $\lambda$ and the eigenvalues $\lambda_1\geq\ldots\geq\lambda_n$ of $L$. Assume $\lambda$ is given such that $\lambda_1>n\lambda>\lambda_{n-1}$.
    
    \sol{\input{laplacian_GD_solutions/9.tex}}
    \item {[Algebraic expression and justification.]}
    Find the number of time steps gradient descent takes to converge to some $\varepsilon>0$ around $x^*$ as a function of $\eta$, assuming $\|x_0-x^*\|_2>\varepsilon$ (since we are not lucky enough to start gradient descent near the optimal point). That is, find $t(\eta)$ such that $\|x_{t(\eta)}-x^*\|_2\leq \varepsilon$. Express $t(\eta)$ in terms of $x_0$, $x^*$, $\lambda$, and $\lambda_1,\ldots,\lambda_n$. Note that $t(\eta)$ should be an integer-valued function.
    
    \sol{\input{laplacian_GD_solutions/10.tex}}
    \item {[Algebraic expression and justification.]}
    To maximize our rate of convergence, we find the optimal step size $\eta^*=\argmin_{\eta>0}t(\eta)$. Express $\eta^*$ and $t(\eta^*)$ in terms of $\lambda,\lambda_1,\ldots,\lambda_n$.
    
    
    \sol{\input{laplacian_GD_solutions/11.tex}}
    \item {[A few sentences.]}
    It turns out the ratio $\kappa=\lambda_1/\lambda_{n-1}$ is inversely related to how connected the graph $G$ is. For example, highly connected graphs such as the complete graph $K_n$ or the hypercube have relatively small values of $\kappa$, while poorly connected graphs such as the 2D grid or a single path have large values of $\kappa$. (If interested, look into conductance of a graph and Cheeger's inequality). Assuming we choose the optimal step size $\eta^*$, how does the connectivity of the graph affect the rate of convergence? Do you expect gradient descent for graphs with higher values of $\kappa$ to converge slower or faster than that for graphs with lower $\kappa$?
   
    \textit{Hint:} The function $f(x)=\dfrac{x-1}{x+1}$ is monotonically increasing for $x\geq 1$.
    
    \sol{\input{laplacian_GD_solutions/12.tex}}
    
\end{enumerate}