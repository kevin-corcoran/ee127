\qns{Jupyter Notebook Setup}

% \todo[inline]{Make sure this is fair for Windows users. 
% Questions in red.
% Answers in blue.
% Merge the two pdfs into one for the submission.
% Add opencv, ioimage, sklearn.
% Debug the Conda environment (work on windows).
% Re-explain the Jupiter notebook submission}

\textbf{Note: Please feel free to ask for help from the TAs during office hours and homework parties to ensure your conda environment works as it should}


\textbf{Conda Download}

If you already, have anaconda, skip to the Conda Environment header. The goal of this problem is to confirm that you are proficient with the software environment, which you will need to complete the class.

\begin{enumerate}
    
    \item To get started, download anaconda from this link: \url{https://www.anaconda.com/distribution/}
    \item Download the command line installer
    \item Run the downloaded script
    \item Follow the instructions on the terminal
    \item Quit the terminal and restart it --- this refreshes the environment and lets your terminal see Conda
\end{enumerate}

\textbf{Conda Environment}
\newline
To create the environment we will be using in this class, run the following command: \newline
\verb|conda env create -f ee127_***.yml|, where \verb|***| corresponds to your system (mac, linux, or windows).

For windows users, you may need to download Visual Studio. If for any reason, the environment creation fails, please refer to \href{https://docs.google.com/document/d/1jwnYQF0PQSmBKwjOcjl1lVbTwcTOevHr2lW5NS8rsC4/edit?usp=sharing}{this link} for instructions on how to manually install the correct libraries.

Then, run \verb|conda activate ee127|. \textbf{You will need to run this command every time you start doing any code work for this class}

Please refer to the corresponding jupyter notebook for the rest of the question: \verb|intro_to_jupyter.ipynb|