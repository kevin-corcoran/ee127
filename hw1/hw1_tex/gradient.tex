\qns{Gradients and Hessians}\\
The \textit{gradient} of a scalar-valued function $g \; : \; \mathbb{R}^n \rightarrow \mathbb{R}$, is the column vector of length $n$, denoted as $\nabla g$, containing the derivatives of components of $g$ with respect to the variables:
\begin{align*}
(\nabla g(x))_i = \dfrac{\partial g}{\partial x_i} (x), \; i = 1,\hdots n.
\end{align*}

The \textit{Hessian} of a scalar-valued function $g \::\: \Real{n} \rightarrow \mathbb{R}$, is the $n \times n$ matrix, denoted as $\nabla^2 g$, containing the second derivatives of components of $g$ with respect to the variables:
\[
(\nabla^2 g(x))_{ij} = \frac{\partial^2 g}{\partial x_i \partial x_j}(x), \;\; i=1,\ldots,n, \;\; j=1,\ldots,n.
\]

For the remainder of the class, we will repeatedly have to take gradients and Hessians of functions we are trying to optimize. This exercise serves as a warm up for future problems.

Compute the gradients and Hessians for the following functions:

\begin{enumerate}
\item $g(x) = y^\top A x$ 

\sol{\item $g(x) = y^\top A x$ }

\item $g(x_1,x_2) = \dfrac{x_1^2}{4} + \dfrac{x_2^2}{9}$




\sol{\item $g(x_1,x_2) = \dfrac{x_1^2}{4} + \dfrac{x_2^2}{9}$


}

\item Show that the following inequalities hold for any vector $x \in \Real{n}$:
\[
% \|x\|_\infty \le \|x\|_1 \le n \|x\|_\infty \\
% \|x\|_2 \le \|x\|_1 \le \sqrt{n} \|x\|_2 \\
% \|x\|_\infty \le \|x\|_2 \le\sqrt{n}\|x\|_\infty \\
\frac{1}{\sqrt{n}}\|x\|_2 \leq \|x\|_\infty \leq  \|x\|_2 \leq \|x\|_1 \leq \sqrt{n} \|x\|_2 \le n\|x\|_\infty.
\]

\textit{Hint:} For $\|x\|_1\leq \sqrt{n}\|x\|_2$, how might you express $\|x\|_1$ as the dot product of two vectors? Can you then use the Cauchy-Schwarz inequality to bound this dot product?

\sol{\item Show that the following inequalities hold for any vector $x \in \Real{n}$:
\[
% \|x\|_\infty \le \|x\|_1 \le n \|x\|_\infty \\
% \|x\|_2 \le \|x\|_1 \le \sqrt{n} \|x\|_2 \\
% \|x\|_\infty \le \|x\|_2 \le\sqrt{n}\|x\|_\infty \\
\frac{1}{\sqrt{n}}\|x\|_2 \leq \|x\|_\infty \leq  \|x\|_2 \leq \|x\|_1 \leq \sqrt{n} \|x\|_2 \le n\|x\|_\infty.
\]

\textit{Hint:} For $\|x\|_1\leq \sqrt{n}\|x\|_2$, how might you express $\|x\|_1$ as the dot product of two vectors? Can you then use the Cauchy-Schwarz inequality to bound this dot product?}

\item $g(x) = \sin(x_1^2) \log (x_3 - x_2)$ where $x_i$ are scalars and $x_3 - x_2 > 0$.

\sol{\item $g(x) = \sin(x_1^2) \log (x_3 - x_2)$ where $x_i$ are scalars and $x_3 - x_2 > 0$.}
\end{enumerate}


% Consider the case now where all vectors and matrices above are scalar; do your answers above make sense? (No need to answer this in your submission)