\qns{Dual of a QP and differentiability}\rm

Consider a quadratic program of the form
\[
p^* = \max_x \: c{\tran} x - \frac{1}{2}x{\tran} Qx ~:~ Ax \le b,
\]
where $A \in \Real{m\times n}$, $b \in \Real{m}$, $c \in \Real{n}$, and $Q \in \Sympp{n}$. We assume that the problem is feasible.
\begin{enumerate}
	\item {[A few lines]}Form the dual of the QP. Show that strong duality holds.
	
	\sol{\item Let $x,y\in\Real{n}$ be two unit-norm vectors, that is, such that $\|x\|_2 = \|y\|_2=1$. Show algebraically that the vectors $x-y$ and $x+y$ are orthogonal. Then, show this graphically by drawing the two vectors on the 2D plane, as well as any other necessary shapes. You may use right angles, circles and straight lines to make your point.}
	\item 
	{[A few lines of justification]}
	Show that $p^*$, considered as a function of $c$ (resp.\ $b$), is convex (resp.\ concave).
	
	\sol{\item Show that the following inequalities hold for any vector $x \in \Real{n}$:
\[
% \|x\|_\infty \le \|x\|_1 \le n \|x\|_\infty \\
% \|x\|_2 \le \|x\|_1 \le \sqrt{n} \|x\|_2 \\
% \|x\|_\infty \le \|x\|_2 \le\sqrt{n}\|x\|_\infty \\
\frac{1}{\sqrt{n}}\|x\|_2 \leq \|x\|_\infty \leq  \|x\|_2 \leq \|x\|_1 \leq \sqrt{n} \|x\|_2 \le n\|x\|_\infty.
\]

\textit{Hint:} For $\|x\|_1\leq \sqrt{n}\|x\|_2$, how might you express $\|x\|_1$ as the dot product of two vectors? Can you then use the Cauchy-Schwarz inequality to bound this dot product?}
	\item 
	{[1-2 lines]}
	{[A few lines of justification]}
	For any convex function $f: \mathbb{R}^n \rightarrow \mathbb{R}$, we say the vector $g \in \mathbb{R}^n$ is a subgradient of $f$ at $x \in \text{\textbf{dom} } f$ if for all $z \in \text{\textbf{dom} } f, f(z) \geq f(x) + g^{\top}(z-x)$.  The subgradient of $f$ at $x$ need not  be unique. The set of all subgradients of $f$ at $x$ is known as the subdifferential and is denoted by $\partial f(x)$. 
	
	Suppose $f$ is the pointwise maximum of subdifferentiable (subdifferential exists at every point) convex functions $f_1, f_2, \cdots f_m$, so that $f(x) = \underset{i}{\text{max}} f_i(x)$. Then, a subgradient of $f$ at $x$ is any vector $g$ such that $g \in \partial f_{k}$ where $k$ is any index such that $f_{k}(x) = f(x)$. In words, to find a subgradient of the maximum of functions at a point, we can choose one of the functions that achieves the maximum at that point, and choose any subgradient of that function at the point. This follows from the fact that $f(z) \geq f_{k}(z) \geq f_{k}(x) + g^{\top}(z-x) = f(x) + g^{\top}(z-x)$. (See section $8.2.3$ in the course text for more details on subgradients).
	
	
	
	Explain how to form a subgradient of $p^*$ considered as a function of $c$. You may assume subdifferentiability wherever it is helpful.
	
	\sol{\item $g(x) = f(Ax + b) x$ where $f: \mathbb{R}^n \mapsto \mathbb{R}$ is once differentiable and $A \in \mathbb{R}^{n \times n}$.}
	
	\item {[1-2 lines]}
	The convex function $f$ is differentiable iff the subgradient at every point is unique. 
	
	
	Is $p^*$ differentiable with respect to $c$?
	
	\sol{\item
\[
g(x) = \begin{bmatrix}
x_1^2/x_2 \\
\log(x_3) \sin(x_1/x_3)
\end{bmatrix}
\]}
\end{enumerate}
